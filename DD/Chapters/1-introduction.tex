\section{Purpose}
As the demand for skilled professionals grows, facilitating smooth connections between students and companies becomes increasingly vital. Traditionally, students seeking internships often face challenges in finding opportunities that align with their skills and career goals, while companies struggle to identify the right talent. The Students\&Companies (S\&C) platform addresses this challenge by providing a space where university students and companies offering internships can easily find and connect with each other.

S\&C simplifies the process by matching students with internships based on their skills, experiences, and the opportunities offered by companies. The platform enhances the internship search and recruitment process, making it more efficient for both students looking to gain real-world experience and companies seeking fresh talent. By fostering better connections and streamlining communication, S\&C helps ensure that both students and companies can find the right fit, ultimately benefiting the growth of the workforce.
\section{Scope}
In this section, we define the domain of the Students\&Companies (S\&C) platform, focusing on the main users and their interactions with the system. There are two primary user groups: Students and Companies.

Students are individuals enrolled in universities who use the platform to search for and apply to internship opportunities. They can create and manage their profiles, upload their CVs, and receive recommendations for internships that match their skills and preferences. Students can also proactively browse available internships, apply to them, and track the status of their applications.

Companies are organizations offering internships, using the platform to advertise available opportunities. They can post detailed internship descriptions, including tasks, benefits, and the skills required. Companies can also review student profiles, receive recommendations, and initiate the selection process, which may involve interviews and other assessments.

The platform facilitates a two-way interaction: students receive recommendations for internships that align with their qualifications, and companies can identify and reach out to students who fit their needs. Both students and companies can track the progress of their applications or listings and provide feedback to improve the matching process. Additionally, universities are involved in monitoring the internships, ensuring quality, and supporting students throughout the experience.
\section{Definitions, Acronyms, Abbreviations}
\begin{itemize}
    \item S\&C Students\&Companies
    \item UI User Interface
    \item RASD Requirement Analysis and Specification Document
    \item DD Design Document
    \item HTTPS/REST The usage of a Representational State Transfer API to commuicate using the HyperText Transfer Protocol Secure protocol
\end{itemize}
\section{Revision history}
Version 1.0 (30/12/2024)
\section{Reference Documents}
This document is based on the following materials
\begin{itemize}
    \item R\&DD assignment specification of the Software Engineering II course a.a. 2024/5
    \item Slides of the same course available on WeBeep
\end{itemize}
\section{Document Structure}
\begin{itemize}
  \item \textbf{Introduction:} In this section we introduce the project and the Design Document.
  \item \textbf{Architectural Design:} This section describes in detail the architecture selected for the project.
  \item \textbf{User Interface Design:} In this section we show some prototypes of the user interface that aim to showcase the look and feel of the platform.
  \item \textbf{ Requirements Traceability:} In this section the requirements defined in the RASD are mapped to the different components defined in this document.
  \item \textbf{Implementation, Integration and Test Plan:} In this section we define the implementation and testing plans, providing a rationale for each.
  \item \textbf{Effort Spent:} Time spent on each section of the document.
  \item \textbf{References:} List of documents and software references used in the paper.
  \end{itemize}

