% This document has been prepared to help you approaching Latex as a formatting tool for your Travlendar+ deliverables. This document suggests you a possible style and format for your deliverables and contains information about basic formatting commands in Latex. A good guide to Latex is available here \href{https://tobi.oetiker.ch/lshort/lshort.pdf}{https://tobi.oetiker.ch/lshort/lshort.pdf}, but you can find many other good references on the web. 

% Writing in Latex means writing textual files having a \texttt{.tex} extension and exploiting the Latex markup commands for formatting purposes. Your files then need to be compiled using the Latex compiler. Similarly to programming languages, you can find many editors that help you writing and compiling your latex code. Here \href{https://beebom.com/best-latex-editors/}{https://beebom.com/best-latex-editors/} you have a short oviewview of some of them. Feel free to choose the one you like.  

% Include a subsection for each of the following items:
% \begin{itemize}
% \item
% Purpose: here we include the goals of the project
% \item
% Scope: here we include an analysis of the world and of the shared phenomena
% \item
% Definitions, Acronyms, Abbreviations
% \item
% Revision history
% \item
% Reference Documents 
% \item
% Document Structure
% \end{itemize}
% Below you see how to define the header for a subsection.


\subsection{Purpose}
As the demand for skilled professionals grows, facilitating smooth connections between students and companies becomes increasingly vital. Traditionally, students seeking internships often face challenges in finding opportunities that align with their skills and career goals, while companies struggle to identify the right talent. The Students\&Companies (S\&C) platform addresses this challenge by providing a space where university students and companies offering internships can easily find and connect with each other.

S\&C simplifies the process by matching students with internships based on their skills, experiences, and the opportunities offered by companies. The platform enhances the internship search and recruitment process, making it more efficient for both students looking to gain real-world experience and companies seeking fresh talent. By fostering better connections and streamlining communication, S\&C helps ensure that both students and companies can find the right fit, ultimately benefiting the growth of the workforce.
\subsubsection{Goals}
\begin{enumerate}[label={[G\arabic*]}]
    \item Students are able to look for internships and easily recognize the ones that matches their characteristics (skills, interests, experience).
    \item Companies are able to advertise their open internship projects and easily asses candidates' fit with them.
    \item Companies and students can have a well-defined, precise and concrete selection process.
    \item Through tailored recommendations, students can find suitable internships when those exist.
    \item Through tailored recommendations, companies can identify fitting candidates when those exist.
\end{enumerate}

\subsection{Scope}
In this section, we define the domain of the Students\&Companies (S\&C) platform, focusing on the main users and their interactions with the system. There are two primary user groups: Students and Companies.

Students are individuals enrolled in universities who use the platform to search for and apply to internship opportunities. They can create and manage their profiles, upload their CVs, and receive recommendations for internships that match their skills and preferences. Students can also proactively browse available internships, apply to them, and track the status of their applications.

Companies are organizations offering internships, using the platform to advertise available opportunities. They can post detailed internship descriptions, including tasks, benefits, and the skills required. Companies can also review student profiles, receive recommendations, and initiate the selection process, which may involve interviews and other assessments.

The platform facilitates a two-way interaction: students receive recommendations for internships that align with their qualifications, and companies can identify and reach out to students who fit their needs. Both students and companies can track the progress of their applications or listings and provide feedback to improve the matching process. Additionally, universities are involved in monitoring the internships, ensuring quality, and supporting students throughout the experience.
\subsubsection{World Phenomena}
\begin{enumerate}[label={[WP\arabic*]}]
    \item Students are interested in internship projects.
    \item Companies have internship positions to fill for projects.
    \item Each student has a set of characteristics (interest, skills, experience) organized in their CV that makes them fit or unfit for certain projects.
    \item Companies have a suited profile for each project and may have customized questionnaires that reflect this suited profile.
    \item Companies interview candidate students to asses their fit for the project.
    \item Once an internship is settled in it prescribes a set of responsibilities (workload, commitment, salary, environment) for both the student and the employer.
    \item Both parts individually keep track of each other's responsibilities and may have complaints about the other part action/behavior.
    \item When the internship ends, both parts may have feedback and suggestion regarding the whole process (interview, selection, development).
\end{enumerate}

\subsubsection{Shared Phenomena}
\begin{enumerate}[label={[SP\arabic*]}]
    \item[] \textbf{World controlled}
    \item Students provide information about their own characteristics.
    \item Companies provide a description, responsibilities and profile of the projects they offer.
    \item Companies upload customized questionnaires that structure their interviews.
    \item Companies submit feedback on interviews.
    \item Users report their complaints to S\&C.
    \item Users provide feedback and suggestions regarding past internships that can be relevant for future recommendations.
    \item[] \textbf{Machine controlled}
    \item Users receive notifications when relevant matching counterparts are available.
    \item During interviews, questionnaires become available to companies to collect students’ answers.
    \item Users communicate through the platform to set up a time and place for the interview.
    \item Students receive the company verdict about the interview process.
\end{enumerate}
\subsection{Definitions, Acronyms, Abbreviations}
\subsection{Revision history}
Version 1.0 (21/12/2024)
Version 1.1 (22/12/2024): added explanation text to the domain class diagram and to the sequence diagrams
Version 2.0 (30/12/2024): following decisions consequent to the design document
\begin{enumerate}
    \item UC11 sequence diagram: Previously, it stated that the student filled the interview questionnaire with their answers. This is incorrect, as we believe the most accurate process would involve the company employee asking the questions to the student and then filling the questionnaire having processed the student's answers (it might involve repeating or explaining questions).
    \item UC7: Previously it started with the user clicking a link on an email notification. While working on the Design Document, we realized there are many ways in which a user can arrive at the ``Closed Internships'' page, not only email notifications.
\end{enumerate}
\subsection{Reference Documents}
This document is based on the following materials
\begin{itemize}
    \item R\&DD assignment specification of the Software Engineering II course a.a. 2024/5
    \item Slides of the same course available on WeBeep
\end{itemize}


\subsection{Document Structure}
\begin{itemize}
  \item \textbf{Introduction:} Brief description of the project, including its purpose, goals, and objectives.
  \item \textbf{Overall Description:} High-level overview of the system, explaining the involved phenomena and domain assumptions.
  \item \textbf{Specific Requirements:} Detailed analysis of requirements, including hardware and software constraints for developers.
  \item \textbf{Formal Analysis:} Formal description of world phenomena using Alloy.
  \item \textbf{Effort Spent:} Time spent on each section of the document.
  \item \textbf{References:} List of documents and software references used in the paper.
\end{itemize}
